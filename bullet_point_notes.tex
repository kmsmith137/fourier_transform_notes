\documentclass[aps,prd,superscriptaddress,groupedaddress,nofootinbib,nobibnotes]{revtex4}

\usepackage{graphicx}
\usepackage{dcolumn}
\usepackage{bm}
\usepackage{amssymb}
\usepackage{epstopdf}
\usepackage{amsmath}
\usepackage{amsfonts}
\usepackage{color}
\usepackage{mathrsfs}
% \usepackage{comment}
% \usepackage{url}
% \usepackage{wick}
% \usepackage{feynmp}
% \usepackage{braket}

\setlength{\parindent}{20pt}
% \setlength{\parskip}{1mm}

\setcounter{topnumber}{1}    % default value is 2.
\setcounter{bottomnumber}{0} % default value is 1.

\hyphenation{ALPGEN}
\hyphenation{EVTGEN}
\hyphenation{PYTHIA}

\newcommand{\kms}[1]{\textcolor{blue}{(KMS: #1)}}
\newcommand{\be}{\begin{equation}}
\newcommand{\ee}{\end{equation}}
\newcommand{\ba}{\begin{eqnarray}}
\newcommand{\ea}{\end{eqnarray}}
\newcommand{\nn}{\nonumber}

\def\tx{{\tilde x}}
\def\bigoh{{\mathcal O}}

\renewcommand{\baselinestretch}{1.1}

\begin{document}

\title{Fourier transform notes}

\author{Kendrick~M.~Smith}
\affiliation{Perimeter Institute for Theoretical Physics, Waterloo, ON N2L 2Y5, Canada}

\date{\today}

% \begin{abstract}
% ABSTRACT HERE
% \end{abstract}
% \pacs{}

\maketitle

\section{The complex discrete Fourier transform (cDFT)}

\begin{itemize}
\item There are many variants of ``the'' Fourier transform, depending on whether the function
 being transformed is periodic/nonperiodic, discretely/continuously sampled, real/complex, etc.
 In this section we'll introduce the simplest variant: the {\em periodic, discrete, complex} transform.
 This variant is sometimes called the ``cDFT'' or just ``DFT''.
\item The input to the cDFT will be a complex sequence with period $N$, i.e.~a sequence of
 complex numbers $x_0, x_1, x_2, \cdots$ such that $x_{j+N} = x_j$ for every integer $j$.
 Such a sequence is uniquely represented by its first $N$ elements $x_0, \cdots, x_{N-1}$,
 and in software we would use a length-$N$ complex array.
\item A specific example of a period-$N$ complex sequence is the ``sinusoid''
\be
 x_j = e^{(2\pi i / N) j k}   \label{eq:sinusoid}
\ee
 where $k$ is an integer and $i=\sqrt{-1}$.  (Note that according to this definition, the
 constant sequence $x_j = 1$ is considered to be a sinusoid by taking $k=0$ in Eq.~(\ref{eq:sinusoid}).)
\item Exercise: show that the sinusoid in Eq.~(\ref{eq:sinusoid}) has period $N$ as claimed.
 Show also that there is a periodicity in $k$: if $k$ and $k'$ differ by an integer multiple of $N$,
 then $x_j = x'_j$ for all integers $j$.
\item More generally, we can construct a period-$N$ complex sequence by taking a linear combination
 of sinusoids: $x_j = \alpha (1) + \beta e^{(2\pi i/N) j} + \gamma e^{(4\pi i/N) j} + \cdots$,
 where $\alpha, \beta, \gamma, \cdots$ are complex coefficients.
\item Let's write the most general linear combination of sinusoids as:
\be
 x_j = \sum_{k=0}^{N-1} \tx_k e^{(2\pi i / N) j k}  \label{eq:DFT1}
\ee
 where $\tx_k = \tx_0, \tx_1, \cdots \tx_{N-1}$ is a sequence of complex coefficients.
\item Exercise: explain why Eq.~(\ref{eq:DFT1}) is the ``most general linear combination of sinusoids''
 even though the sum only runs from $k=0$ to $N-1$.  Wouldn't we get a more general form by taking
 the sum from $k=0$ to $N$ (say)?
\item There is a very important theorem which says that any period-$N$ complex sequence $x_j$
 can be represented in the form in Eq.~(\ref{eq:DFT1}), for a suitable choice of coefficients $\tx_k$
 (its ``Fourier coefficients'').
 In other words, an arbitrary sequence $x_j$ is the sum of a series of sinusoids (the ``Fourier series'').
 All of Fourier analysis is based on this theorem!  Proof omitted from notes for now, but I'll
 write it on the blackboard.
\item Another fundamental result is a formula which allows the coefficients $\tx_k$ in Eq.~(\ref{eq:DFT1})
 to be computed from the sequence $x_j$.  The formula is:
\be
  \tx_k = \frac{1}{N} \sum_{j=0}^{N-1} x_j e^{-(2\pi i / N) j k}  \label{eq:DFT2}
\ee
 Now we have two formulas: Eq.~(\ref{eq:DFT2}) which allows the Fourier coefficients $\tx_k$ to be
 computed from a sequence $x_j$, and Eq.~(\ref{eq:DFT1}) which allows $x_j$ to be computed from $\tx_k$.
 The pair of operations $\tx \rightarrow x$ and $\tx \rightarrow x$ are called the ``Fourier transform''
 and the ``inverse Fourier transform'' respectively.
 Note that the Fourier transform and its inverse are nearly the same algebraic operation: 
 the only difference is the replacement $i \rightarrow (-i)$ and the prefactor $(1/N)$.
\item Exercise: Write a python function to compute the Fourier transform $\tx \rightarrow x$ as defined in Eq.~(\ref{eq:DFT1}),
 and a function to the inverse Fourier transform as defined in Eq.~(\ref{eq:DFT2}).  For now, write these functions
 ``by hand'' rather than using python's built-in Fourier transform routines.  Both functions should accept
 a length-$N$ numpy array and return a length-$N$ numpy array.  Verify that the two functions are inverses of each
 other, when applied to random input arrays.
\item The reader should note that different authors define the Fourier transform and its inverse slightly differently.
 We have put the $(+i)$ in the Fourier transform and the $(-i)$ in the inverse transform~(\ref{eq:DFT2}),
 but sometimes the opposite convention is used.  We have put the prefactor $(1/N)$ in the inverse transform,
 and prefactor 1 in the Fourier transform, but the opposite convention can be used.  (Another possible convention is
 to use prefactor $1/\sqrt{N}$ in both transforms.)
\item Exercise: Python has built-in Fourier transforms {\tt numpy.fft.fft()} and {\tt numpy.fft.ifft()}.
 By playing with these routines, figure out what Fourier conventions are used in python.
 How would you ``doctor'' these routines to agree with the Fourier conventions used in these notes?
\item Exercise: compare the running time of your ``homegrown'' Fourier transform routines above to python's
 builtin Fourier transforms, for a few large values of $N$ (say $N=1024, 2048, \cdots$).  You should find that
 python's transforms are much faster.  This is because there is a non-obvious algorithm (the ``fast Fourier
 transform'' or FFT) for speeding up the Fourier transform.  The FFT is sometimes said to be the most important
 algorithm in science!  A useful thing to know is that the FFT is particularly fast when $N$ is a power of two,
 or more generally a product of small primes (although its asymptotic complexity is $\bigoh(N \log N)$ for all $N$).
\item Exercise: show from the definition that the FFT is a linear operation, i.e.~if $z_j = \alpha x_j + \beta y_j$,
 where $\alpha,\beta$ are complex numbers and $f_j,g_j$ are periodic sequences, then 
 $\tilde z_j = \alpha \tilde x_j + \beta \tilde y_j$.  (This is sometimes called the ``addition theorem''.)
\end{itemize}

%\begin{figure}
%\centerline{\includegraphics[width=14cm]{x.pdf}}
%\caption{xxx}
%\label{fig:xxx}
%\end{figure}

% Note: Syntax for clipping a figure is
% \includegraphics[trim={5cm 0 0 0},clip,...]
%  where ordering is <left> <lower> <right> <upper>

% \section*{Acknowledgments}
%
% Research at Perimeter Institute is supported by the Government of Canada
% through Industry Canada and by the Province of Ontario through the Ministry of Research \& Innovation.
% Some computations were performed on the GPC cluster at the SciNet HPC Consortium.
% SciNet is funded by the Canada Foundation for Innovation under the auspices of Compute Canada,
% the Government of Ontario, and the University of Toronto.
% KMS was supported by an NSERC Discovery Grant and an Ontario Early Researcher Award.

% \bibliographystyle{h-physrev}
% \bibliography{xxx}

% \appendix
% \section{Appendix}

\end{document}
